% Copyright [2022] Michel Hidalgo <hid.michel@gmail.com>
%
% Licensed under the Apache License, Version 2.0 (the "License");
% you may not use this file except in compliance with the License.
% You may obtain a copy of the License at
%
%     http://www.apache.org/licenses/LICENSE-2.0
%
% Unless required by applicable law or agreed to in writing, software
% distributed under the License is distributed on an "AS IS" BASIS,
% WITHOUT WARRANTIES OR CONDITIONS OF ANY KIND, either express or implied.
% See the License for the specific language governing permissions and
% limitations under the License.

% Local Variables:
% coding: utf-8
% mode: latex
% TeX-engine: xetex
% End:

% Avoid intersentence spacing noise
% chktex-file 13
\documentclass[spanish, letterpaper]{resume}

% Override \date with intended style
\usepackage{datetime2}
\renewcommand{\date}[2]{%
  \DTMspanishMonthname{#2} #1}

% Add images to path
\usepackage{graphicx}
\graphicspath{{./images/}}

% Setup PDF metadata
\hypersetup{%
  pdftitle={CV de Michel Hidalgo},
  pdfauthor={Michel Hidalgo},
  pdfsubject={},
  pdfcreator={XeLaTeX},
  pdfproducer={},
  pdfkeywords={},
  unicode=true,
}

\begin{document}

\begin{center}
  \title{Michel Hidalgo} \\
  Ingeniero de Software para Robótica \\
  \address{Buenos Aires}{
    Ciudad Autónoma de Buenos Aires,
    Argentina} \\
  \begin{inlined}
    \item \github{hidmic}
    \item \email{hid.michel@gmail.com}
    \item \linkedin{michel-hidalgo}
  \end{inlined}
\end{center}

\section*{Resumen}

Ingeniero en Electrónica por formación pero ingeniero en Informática de profesión, con más de 10 años de experiencia desarrollando software y más de 8 años enfocado en la robótica. Fascinado por los mecanismos de percepción, actuación y toma de decisión en robótica.

\section*{Conocimientos}

\hfill
\begin{minipage}{0.55\linewidth}
  \begin{description}
    \item[Programación] C++11/14/17, Python 3.x, C99/11.
    \item[Robótica y Simulación] ROS, ROS 2, Drake, Gazebo.
    \item[Cómputo científico] Eigen, SciPy, NumPy, SymPy, R, Matlab.
    \item[Integración continua] Docker, Github Actions, Gitlab CI.
  \end{description}
\end{minipage}\hfill
\begin{minipage}{0.4\linewidth}
  \begin{description}
    \item[Compilación] CMake, Bazel, Colcon.
    \item[Sistemas operativos] Linux, Nuttx.
    \item[Escritura técnica] La\TeX, ReST, Doxygen-like.
    \item[Bilingüe] Inglés (C1), Español (nativo).
  \end{description}
\end{minipage}
\hfill

\section*{Experiencia}

\subsection*[\icon{ekumen}]{Ekumen \at{Buenos Aires, Argentina}}

\subsubsection*{Ingeniero Senior de Software \thru{\date{2018}{9}}{Present}}
\begin{itemize}
  \item Lidero los esfuerzos de I+D con proyectos sobre localización y mapeo, control no lineal, aprendizaje automático, y más.
  \item Lideré equipos de hasta 5 ingenieros a la vez para clientes en EE.UU.
  \item Colaboré con OSRC en el diseño, implementación y mantenimiento de paquetes básicos de ROS 2.
  \item Trabajé en un robot móvil basado en ROS 2 y un par de drones maritimos para investigación oceanográfica.
  \item Trabajé en recollección de datos y análisis de series de tiempo para monitorear plataformas de CI.
  \item Desarrollé un robot móvil basado en ROS 2 para la industria agro-ganadera de punta a punta.
\end{itemize}

\subsubsection*{Ingeniero de Software \thru{\date{2015}{8}}{\date{2018}{9}}}
\begin{itemize}
  \item Trabajé en librerías en C++ para describir redes de caminos en simulaciones para autos autónomos.
  \item Trabajé en robots móviles basados en ROS para la industria hotelera y gestión de inventarios.
\end{itemize}

\subsection*[\icon{utn.ba}]{Universidad Tecnológica Nacional (UTN.BA) \at{Buenos Aires, Argentina}}

\subsubsection*{Ayudante de cátedra \thru{\date{2019}{2}}{\date{2020}{12}}}
\begin{itemize}
  \item Para \textit{Técnicas Digitales III}, una materia de último año sobre arquitecturas de procesador y desarrollo de kernel.
\end{itemize}

\subsection*[\icon{minit.ar}]{Ministerio del Interior y Transporte \at{Buenos Aires, Argentina}}

\subsubsection*{Ingeniero de Software \thru{\date{2012}{11}}{\date{2015}{8}}}
\begin{itemize}
  \item Trabajé en algunos prototipos para Transporte: una librería en C para POS SUBE, una aplicación HTML5 para pantallas de información en las estaciones, entre otros.
  \item Hice mantenimiento de varios sistemas de gobierno, por ejemplo liquidación de sueldos y subsidios. En todos los casos, el backend era Python2.x; el frontend, PHP; y la base de datos, PostgreSQL.
\end{itemize}

\section*{Educación}
\subsection*[\icon{utn.ba}]{Universidad Tecnológica Nacional (UTN.BA) \at{Buenos Aires, Argentina}}
\subsubsection*{Grado en Ingeniería Electrónica \thru{\date{2012}{03}}{\date{2021}{08}}}
\begin{itemize}
    \item Tesina sobre el diseño e implementación de un robot diferencial para monitoreo de nivel de señal WiFi (no publicada).
\end{itemize}

\noindent\begin{minipage}[t]{0.475\linewidth}
  \section*{Presentaciones}
  \subsection*[\icon{ros}]{ROS World 2021}
  \begin{itemize}
    \item Miembro del comité de revisión del programa.
  \end{itemize}
  \subsection*[\icon{ros}]{ROSCon 2019 Macau}
  \begin{itemize}
    \item Coautor de \href{https://roscon.ros.org/2019/talks/roscon2019_composablenodes.pdf}{Composable nodes in  ROS 2}
    \item Coautor de \href{https://roscon.ros.org/2019/talks/roscon2019_markupextensionsforros2launch.pdf}{Markup Descriptions in ROS 2 Launch}
  \end{itemize}
\end{minipage}\hfill
\begin{minipage}[t]{0.475\linewidth}
  \section*{Proyectos de código abierto}
  \begin{description}
    \item[\githubproject{beluga}{Ekumen-OS/beluga}] Localización por Monte Carlo.
    \item[\githubproject{ltitop}{hidmic/ltitop}] Optimización de topologías LTI (autor).
  \end{description}
\end{minipage}

\end{document}

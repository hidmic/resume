% Copyright [2022] Michel Hidalgo <hid.michel@gmail.com>
%
% Licensed under the Apache License, Version 2.0 (the "License");
% you may not use this file except in compliance with the License.
% You may obtain a copy of the License at
%
%     http://www.apache.org/licenses/LICENSE-2.0
%
% Unless required by applicable law or agreed to in writing, software
% distributed under the License is distributed on an "AS IS" BASIS,
% WITHOUT WARRANTIES OR CONDITIONS OF ANY KIND, either express or implied.
% See the License for the specific language governing permissions and
% limitations under the License.

% Local Variables:
% coding: utf-8
% mode: latex
% TeX-engine: xetex
% End:
\documentclass[english, letterpaper]{resume}

% Override \date with intended style
\usepackage{datetime2}
\renewcommand{\date}[2]{%
  \DTMenglishshortmonthname{#2} #1}

% Add images to path
\usepackage{graphicx}
\graphicspath{{./images/}}

% Setup PDF metadata
\hypersetup{%
  pdftitle={Michel Hidalgo's Resume},
  pdfauthor={Michel Hidalgo},
  pdfsubject={},
  pdfcreator={XeLaTeX},
  pdfproducer={},
  pdfkeywords={},
  unicode=true,
}

\begin{document}

\begin{center}
  \title{Michel Hidalgo} \\
  Robotics Software Engineer \\
  \address{Buenos Aires}{
    Ciudad Autónoma de Buenos Aires,
    Argentina} \\
  \begin{inlined}
    \item \github{hidmic}
    \item \email{hid.michel@gmail.com}
    \item \linkedin{michel-hidalgo}
  \end{inlined}
\end{center}

\section*{Summary}

Electronics engineer by training but software engineer by trade, with 10+ years experience developing software and 8+ years in the robotics scene. Fascinated by perception, actuation, and decision making problems in robotics.

\section*{Skills}

\hfill
\begin{minipage}{0.55\linewidth}
  \begin{description}
    \item[System Programming] C++11/14/17, Python 3.x, C99/11.
    \item[Robotics \& Simulation] ROS, ROS 2, Gazebo, Drake.
    \item[Scientific Computing] Eigen, SciPy, SymPy, R, Matlab.
    \item[Technical Writing] La\TeX, ReST, Doxygen-like.
  \end{description}
\end{minipage}\hfill
\begin{minipage}{0.4\linewidth}
  \begin{description}
    \item[Computer Engineering] Linux, Nuttx.
    \item[Build Engineering] CMake, Bazel, Colcon.
    \item[CI/CD] Docker, Github Actions, Gitlab CI.\@
    \item[Bilingual] English (C1), Spanish (native).
  \end{description}
\end{minipage}
\hfill

\section*{Experience}

\subsection*[\icon{ekumen}]{Ekumen \at{Buenos Aires, Argentina}}

\subsubsection*{Lead Software Engineer \thru{\date{2018}{9}}{Present}}
\begin{itemize}
  \item Leading R\&D efforts through projects on localization and mapping, nonlinear control, machine learning, and more.
  \item Led teams of up to 5 SWEs at a time for US-based customers, from SoW through successful delivery.
  \item Collaborated with late OSRC's ROS 2 team on the design, implementation, and maintenance of ROS 2 core packages.
  \item Worked on a ROS 2-powered AMR prototype and ROS-powered AUVs for oceanographic research, on multiple fronts such as IMU/GPS sensor integration, EKF-based localization, Graph SLAM-based mapping, and simulation using Gazebo and Ignition.
  \item Worked on data collection and timeseries analysis to help triage and monitor a ROS 2-based CI/CD pipeline.
  \item Developed the entire software stack of a ROS 2-powered AMR prototype for the poultry industry from the ground up, from trajectory controllers to perception pipelines.
\end{itemize}

\subsubsection*{Software Engineer \thru{\date{2015}{8}}{\date{2018}{9}}}
\begin{itemize}
  \item Worked on C++ libraries for road network descriptions to support autonomous driving car simulations.
  \item Worked on ROS-powered AMRs for the hospitality industry and for retail inventory management, on multiple fronts such as EKF + AMCL-based localization, navigation using DWA and potential fields, simulation using Gazebo, and deployment via debians.
\end{itemize}

\subsection*[\icon{utn.ba}]{Universidad Tecnológica Nacional (UTN.BA) \at{Buenos Aires, Argentina}}

\subsubsection*{Teaching Assistant \thru{\date{2019}{2}}{\date{2020}{12}}}
\begin{itemize}
  \item For \textit{Técnicas Digitales III}, a final year course on processor architectures and kernel development.
\end{itemize}

\subsection*[\icon{minit.ar}]{Ministerio del Interior y Transporte \at{Buenos Aires, Argentina}}

\subsubsection*{Software Engineer \thru{\date{2012}{11}}{\date{2015}{8}}}
\begin{itemize}
  \item Worked on a few prototypes for the Transport division: a C library to handle custom POS hardware, an HTML5 application for a passenger information screen, among others.
  \item Maintained several government IT systems: payroll, subsidies allocation, among others. In all cases, a Python2.x backend, a PHP frontend, and a PostgreSQL database.
\end{itemize}

\section*{Education}
\subsection*[\icon{utn.ba}]{Universidad Tecnológica Nacional (UTN.BA) \at{Buenos Aires, Argentina}}
\subsubsection*{Engineer's degree in Electronic Engineering \thru{\date{2012}{03}}{\date{2021}{08}}}
\begin{itemize}
    \item Thesis on the design and implementation of a diff-drive robot for WiFi signal strength surveying (not published).
\end{itemize}

\noindent\begin{minipage}[t]{0.475\linewidth}
  \section*{Presentations}
  \subsection*[\icon{ros}]{ROS World 2021}
  \begin{itemize}
    \item Member of the Program Committee.
  \end{itemize}
  \subsection*[\icon{ros}]{ROSCon 2019 Macau}
  \begin{itemize}
    \item Co-author of \href{https://roscon.ros.org/2019/talks/roscon2019_composablenodes.pdf}{Composable nodes in  ROS 2}.
    \item Co-author of \href{https://roscon.ros.org/2019/talks/roscon2019_markupextensionsforros2launch.pdf}{Markup Descriptions in ROS 2 Launch}.
  \end{itemize}
\end{minipage}\hfill
\begin{minipage}[t]{0.475\linewidth}
  \section*{Open Source Projects}
  \begin{description}
    \item[\githubproject{beluga}{Ekumen-OS/beluga}] An MCL library (contributor).
    \item[\githubproject{ltitop}{hidmic/ltitop}] LTI optimization toolkit (author).
  \end{description}
\end{minipage}

\end{document}
